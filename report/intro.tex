\section{Introduction}
Every day, billions of emails are sent across the internet using the Simple Mail
Transfer Protocol (SMTP).  When it was originally devised in 1982, SMTP did not
make any specifications regarding encryption and security -- these details were
only added many years later, as part of Extended SMTP (ESMTP).  Yet even today,
encryption is not required for sending emails, and so an email message may
appear in plain-text somewhere in its travels.

We wanted to determine how likely it is that email communication between two 
users is protected against possible eavesdropping.  Recently it has been brought 
to the public’s attention that in addition to many solitary malicious users, 
there are large organisations performing mass surveillance of online 
communications. Many users rely on email for online communication, sometimes 
including for sensitive data such as bank information and passwords. In order 
to protect user's email communication encryption must be used. We set out to 
study how well email encryption is implemented and how many users it protects.
